\documentclass[a4paper,10pt]{article}

\usepackage[dvips,pdftex]{graphicx}
\usepackage{longtable}
\usepackage{a4wide}
\usepackage{color}
\definecolor{red}{rgb}{1,0,0}
\bibliographystyle{alpha}

\title{Bibsonomy Webservice API: Developer's Guide}
\author{Manuel Bork manuel.bork@uni-kassel.de}
% \date{08.10.2006}

\begin{document}
\maketitle
% \tableofcontents
\begin{abstract}
Bibsonomy\footnote{http://www.bibsonomy.org} is a social bookmark and publication sharing system developed on the chair of knowledge and data engineering\footnote{http://www.kde.cs.uni-kassel.de} of the university of Kassel ~\cite{hotho2006bibsonomy} ~\cite{hotho2006emergent}. The available document is a guidance how to develop applications using the Bibsonomy webservice API.
\end{abstract}

\section{Introduction}
Bibsonomy provides a webservice using \textit{Representational State Transfer} (REST), a software architectural style for distributed hypermedia systems. The term originated in a 2000 doctoral dissertation about the web written by Roy Fielding~\cite{fielding00}, one of the principal authors of the HTTP protocol specification. This article is intended for developers who want to develop applications which interact with Bibsonomy.

\section{The Java client library}
Though it's possible to directly interact with Bibsonomy's webservice, as described in section \ref{direct}, the recommended way is to use the provided client library, written in Java. The Java library encapsulates each accessing method provided by the webservice (see appendix \ref{urlPatterns}) with a corresponding class. 

\subsection{Getting started}
The client library is free for download via HTTP from  \textcolor{red}{http://www.bibsonomy.org/...}. The client library consists of multiple jars which must be added to your Java project's classpath. Those jars are:
\begin{description}
 \item[Bibsonomy webservice] The webservice client itself consists of three jars: \textcolor{red}{bibsonomy-client.jar} (client library), \textcolor{red}{bibsonomy-data-model.jar} (bibsonomy data model), \textcolor{red}{bibsonomy-shared.jar} (shared library).
 \item[Apache HTTP-Client] The HTTP client libraries by the apache software foundation are used for the HTTP communication stuff\footnote{http://jakarta.apache.org/commons/httpclient/}: commons-httpclient.jar.
 \item[Java Architecture for XML Binding (JAXB)] The client uses JAXB to (de)serialize the XML exchange documents\footnote{http://java.sun.com/webservices/jaxb/}: jaxb-api.jar.jar, jaxb-impl.jar, jsr173.jar, activation.jar
 \item[Fujaba associations] This library is used by the data model for encapsulating the n:n and n:m associations\footnote{http://www.fujaba.de}: assocs.jar. 
 \end{description}


\subsection{Using the client library}
Using the client library is rather simple. At first one has to instantiate an object of type \textit{Bibsonomy} and set the user's credentials on it. This object can be reused during program run. As already discussed, there exist classes encapsulating each provided webservice method. These classes are derived from \textit{AbstractQuery} and request in their constructor parameters (usually model data objects, see the package \textcolor{red}{\textit{org.bibsonomy.model}}, and see figure \ref{fig:datamodel} for the uml class diagram of the Bibsonomy data model) - please refer to the Java doc of the corresponding constructor. After instantiating an appropriate \textit{AbstractQuery} just execute it by invoking the \textit{executeQuery} method of the Bibsonomy object with the query object as parameter. If your query object fetches a list of objects, for example a list of posts, you can also use the overloaded \textit{executeQuery} method and register a callback object (use the \textit{ProgressCallback} interface). The return values of your query are available via the \textit{getResult} and \textit{getHttpStatusCode} methods of the \textit{AbstractQuery}.
\begin{figure}
 \centering
 \includegraphics[scale=0.5]{datamodel.png}
 % datamodel.png: 1179666x1179666 pixel, 0dpi, infxinf cm, bb=
 \caption{UML class diagram of the Bibsonomy data model}
 \label{fig:datamodel}
\end{figure}

Here is a simple example of how to use the client library:

\begin{verbatim}

 // create webservice client object
 Bibsonomy bib = new Bibsonomy( "myUserName", "mySecretPassword" );

 // instantiate query object
 GetPostsQuery gpq = new GetPostsQuery();
 // some queries can be parametrized
 // in this example we want to fetch only bibtex entries
 gpq.setResourceType( ResourceType.BIBTEX );
 
 try
 {
    // perform query
    bib.executeQuery( gpq );

    // on success, read results
    if( gpq.getHttpStatusCode() == 200 )
    {
       List<Post> posts = gpq.getResult();
       for( Post post: posts )
       {
          /*
           * now do something with your posts :)
           */
       }
    }
 }
 catch( ErrorPerformingRequestException e )
 {
    /*
     * happens on network failure for example
     */
 }
\end{verbatim}

A whole example application for managing bookmarks and publications, demonstrating the webservice client library, is available for download at \textcolor{red}{http://www.bibsonomy.org/...}.

\section{Direct use of the REST webservice}
\label{direct}
If you intend to use the Bibsonomy webservice with another programming language than Java, you are welcome to develop your own client library communicating directly with the REST webservice. In the REST architecture's philosophy each accessible object gets its own URI. The accessible objects of Bibsonomy are \textit{users}, \textit{groups}, \textit{posts} and \textit{tags}. See ~\cite{hotho2006bibsonomy} for an introduction into their semantics. The exposed URIs are summarized in appendix \ref{urlPatterns}. 
REST webservices exchange structured data; so all exchange documents must be valid to a given XML schema. Appendix \ref{xmlPattern} shows an extract of this schema, it can be viewed in whole on \textcolor{red}{https://www.bibsonomy.org/BibWiki/RestApiUrls \textbf{TODO: move documentation to a public website!}}.
If you succeed in writing your own client library, please let us know and send us your files, so that other programmers can benefit from your work. Thank you in advance.

\begin{appendix}
\section{URL pattern}
\label{urlPatterns}
For the uniform addressing and identification a URL pattern was specified, which one can introduce oneself like a file system. The individual inquiries can be partly supplemented with attributes of the form \textit{key=value}. 

\ttfamily
\fontsize{9}{1}
\begin{longtable}{lll}
\textbf{Description} & \textbf{HTTP-Method} & \textbf{URL} \\
\endhead
List of all existing users & GET & /users \\ 
Create user & POST & /users \\ 
An user's details & GET & /users/[username]
 \\ 
Change an user's details & PUT & /users/[username]
 \\ 
Delete an user & DELETE & /users/[username]
 \\ 
List of a user's posts & GET & /users/[username]/posts \\
&& ?tags=[t1+t2+...+tn]\\
&& ?resourcetype=(bibtex|bookmark)
\\
Add a new post & POST & /users/[username]/posts \\ 
A post's details & GET & /users/[username]/posts/[resourceHash] \\ 
Change of a post & PUT & /users/[username]/posts/[resourceHash] \\ 
Delete a post & DELETE & /users/[username]/posts/[resourceHash]
\\
&&\\
List of all groups & GET & /groups \\ 
Add a new group & POST & /groups \\ 
A group's details & GET & /groups/[groupname] \\ 
Change a group's details & PUT & /groups/[groupname] \\ 
Delete group & DELETE & /groups/[groupname] \\ 
List of a group's members & GET & /groups/[groupname]/users \\ 
Add a group member & POST & /groups/[groupname]/users \\ 
Remove a group member & DELETE & /groups/[groupname]/users/[username]
\\
List of all tags & GET & /tags \\ 
 &  & ?filter=[regex] \\ 
 &  & ?(user|group|viewable)=[username/groupname] \\ 
A tag's details & GET & /tags/[tag] \\
Substitute all occurrences of & PUT & /substitutetags?from=[t1+t2+..]\&to=[T1+T2+..]\\
t1,t2,.. by T1,T2,..  \\ 
 &  &  \\ 
List of all posts & GET & /posts \\ 
 &  & ?tags=[t1+t2+...+tn] \\ 
 &  & ?resourcetype=(bibtex|bookmark) \\ 
 &  & ?(user|group|viewable)=[username/groupname] \\ 
 &  & ?resource=[hash] \\ 
List of all new posts & GET & /posts/added \\ 
 &  & ?resourcetype=(bibtex|bookmark) \\ 
List of all popular posts & GET & /posts/popular
\end{longtable}
\rmfamily

All URLs can be supplemented by these attributes (use \textit{start} and \textit{end} only with lists):

\ttfamily
\begin{center}
\begin{tabular}{l}
 ?format=(xml|rdf|html)\\
 ?start=[int], starting at 0, default 0\\
 ?end=[int], starting at 0, default 19
\end{tabular}
\end{center}
\rmfamily
\textit{Tags} can be customized the following way:
\ttfamily
\begin{center}
\begin{tabular}{ll}
 ->[tag] & tag and its direct children \\
 -->[tag] & tag and its children (transitive)\\
\lbrack tag]-> & tag and its direct parents\\
\lbrack tag]--> & tag and its parents (transitive)\\
 <->[tag] & tag and its correlated tags
\end{tabular}
\end{center}
\rmfamily

\section{XML schema}
\label{xmlPattern}
For data exchange XML documents are used. The individual exchange documents are valid to a XML schema, which is represented for the \textit{posts} in part here. The complete schema with examples and explanations can be viewed at \textcolor{red}{https://www.bibsonomy.org/BibWiki/RestApiUrls \textbf{TODO: move documentation to a public website!}}.

\fontsize{9}{1}
\begin{verbatim}
<!-- a post -->
<xsd:complexType name="PostType">
  <xsd:sequence>
    <xsd:element name="user" type="UserType"/>
    <xsd:element name="group" type="GroupType" minOccurs="0" maxOccurs="unbounded"/>
    <xsd:element name="tag" type="TagType" maxOccurs="unbounded"/>
    <xsd:choice>
      <xsd:element name="bookmark" type="BookmarkType"/>
      <xsd:element name="bibtex" type="BibtexType"/>
    </xsd:choice>
   </xsd:sequence>
   <xsd:attribute name="description" type="xsd:string"/>
</xsd:complexType>

<!-- a bookmark -->
<xsd:complexType name="BookmarkType">
  <xsd:attribute name="url" type="xsd:anyURI" use="required"/>
</xsd:complexType>

<!-- a bibtex -->
<xsd:complexType name="BibtexType">
  <xsd:attribute name="title" type="xsd:string" use="required"/>
  <xsd:attribute name="authors" type="xsd:string"/>
  <xsd:attribute name="editors" type="xsd:string"/>
  <xsd:attribute name="type" type="xsd:string"/>
  <xsd:attribute name="year" type="xsd:string"/>
</xsd:complexType>

<!-- a user -->
<xsd:complexType name="UserType">
  <xsd:attribute name="name" type="xsd:string" use="required"/>
</xsd:complexType>

<!-- a tag -->
<xsd:complexType name="TagType">
  <xsd:attribute name="name" type="xsd:string" use="required"/>
</xsd:complexType>
\end{verbatim} 
\end{appendix}
\bibliography{references}
\end{document}